\documentclass{ctexart}
\usepackage{anyfontsize}
\usepackage{hyperref}
\usepackage{graphicx}
\usepackage{amsmath,amsthm}

\author{夏海淞}
\title{线性代数笔记}
\date{\today}

\ctexset { section = { name={第,章} } }
\ctexset { section = { number={\chinese {section}} } }

\newtheorem{theorem}{定理}[subsection]
\newtheorem{lemma}{引理}[subsection]
\newtheorem{definition}{定义}[subsection]
\newtheorem{property}{性质}[subsection]
\newtheorem{infer}{推论}[subsection]

\newcommand{\bsym}[1]{\boldsymbol{#1}}
\newcommand{\mypar}[1]{\left( #1 \right)}
\newcommand{\func}[2]{\mathrm{#1}\mypar{#2}}
\newcommand{\entry}[3]{\func{entry}{#1,#2,#3}}
\newcommand{\row}[2]{\func{row}{#1,#2}}
\newcommand{\col}[2]{\func{col}{#1,#2}}
\newcommand{\trace}[1]{\func{Tr}{#1}}
\newcommand{\diag}[1]{\func{diag}{#1}}
\newcommand{\rank}[1]{\func{rank}{#1}}

\newcommand{\mata}{\bsym{A}}
\newcommand{\matb}{\bsym{B}}
\newcommand{\matc}{\bsym{C}}
\newcommand{\matd}{\bsym{D}}
\newcommand{\mate}{\bsym{E}}
\newcommand{\matf}{\bsym{F}}
\newcommand{\matg}{\bsym{G}}
% \newcommand{\math}{\bsym{H}}
\newcommand{\mati}{\bsym{I}}
\newcommand{\matj}{\bsym{J}}
\newcommand{\matk}{\bsym{K}}
\newcommand{\matl}{\bsym{L}}
\newcommand{\matm}{\bsym{M}}
\newcommand{\matn}{\bsym{N}}
\newcommand{\mato}{\bsym{O}}
\newcommand{\matp}{\bsym{P}}
\newcommand{\matq}{\bsym{Q}}
\newcommand{\matr}{\bsym{R}}
\newcommand{\mats}{\bsym{S}}
\newcommand{\matt}{\bsym{T}}
\newcommand{\matu}{\bsym{U}}
\newcommand{\matv}{\bsym{V}}
\newcommand{\matw}{\bsym{W}}
\newcommand{\matx}{\bsym{X}}
\newcommand{\maty}{\bsym{Y}}
\newcommand{\matz}{\bsym{Z}}

\newcommand{\veca}{\bsym{a}}
\newcommand{\vecb}{\bsym{b}}
\newcommand{\vecc}{\bsym{c}}
\newcommand{\vecd}{\bsym{d}}
\newcommand{\vece}{\bsym{e}}
\newcommand{\vecf}{\bsym{f}}
\newcommand{\vecg}{\bsym{g}}
\newcommand{\vech}{\bsym{h}}
\newcommand{\veci}{\bsym{i}}
\newcommand{\vecj}{\bsym{j}}
\newcommand{\veck}{\bsym{k}}
\newcommand{\vecl}{\bsym{l}}
\newcommand{\vecm}{\bsym{m}}
\newcommand{\vecn}{\bsym{n}}
\newcommand{\veco}{\bsym{o}}
\newcommand{\vecp}{\bsym{p}}
\newcommand{\vecq}{\bsym{q}}
\newcommand{\vecr}{\bsym{r}}
\newcommand{\vecs}{\bsym{s}}
\newcommand{\vect}{\bsym{t}}
\newcommand{\vecu}{\bsym{u}}
\newcommand{\vecv}{\bsym{v}}
\newcommand{\vecw}{\bsym{w}}
\newcommand{\vecx}{\bsym{x}}
\newcommand{\vecy}{\bsym{y}}
\newcommand{\vecz}{\bsym{z}}


\begin{document}
\maketitle
\tableofcontents
\section{矩阵}

\subsection{矩阵和向量的基本概念}

\subsubsection*{矩阵的基本概念}

\subsubsection*{向量的基本概念}

\subsection{矩阵与向量的运算}

\subsubsection*{矩阵(向量)的线性运算}

\subsubsection*{向量的内积与矩阵的乘法}

\begin{property}
    设矩阵\(\mata_{m\times l},\matb_{l\times n}\),则\(\mypar{\mata\matb}^\top=\matb^\top\mata^\top\)。
\end{property}

\begin{property}
    设矩阵\(\mata_{m\times n},\matb_{n\times m}\),则\(\trace{\mata\matb}=\trace{\matb\mata}\)。
\end{property}

\begin{property}
    设\(\mata\)为\(m\times n\)矩阵,\(\mati_m\)和\(\mati_n\)分别为\(m\)阶单位阵和\(n\)阶单位阵,则\(\mati_m\mata=\mata\mati_n=\mata\)。
\end{property}

\begin{property}
    设\(\mata=\diag{a_1,a_2,\dots,a_n}\),\(\matb=\diag{b_1,b_2,\dots,b_n}\),则\(\mata\matb=\matb\mata=\diag{a_1b_1,a_2b_2,\dots,a_nb_n}\)。
\end{property}

\begin{property}
    设\(\mata\)与\(\matb\)均为\(n\)阶上(下)三角阵,则\(\mata\matb\)是上(下)三角阵。
\end{property}

\subsubsection*{方阵的幂}

\subsection{分块矩阵及其运算}

\subsubsection*{分块矩阵}

\subsubsection*{分块矩阵的基本运算}

\subsection{矩阵的初等变换与秩}

\subsubsection*{矩阵的初等变换}

\begin{theorem}
    对矩阵\(\mata_{m\times n}\)作一次初等行变换,相当于在\(\mata_{m\times n}\)的左边乘上一个相应的\(m\)阶初等矩阵;对矩阵\(\mata_{m\times n}\)作一次初等列变换,相当于在\(\mata_{m\times n}\)的右边乘上一个相应的\(n\)阶初等矩阵。
\end{theorem}

\subsubsection*{矩阵的标准形与秩}

\begin{lemma}
    设两个\(m\times n\)矩阵的标准形如下所示
    \begin{equation*}
        \matp_1=\begin{bmatrix}
            \mati_{r_1} & \mato \\\mato&\mato
        \end{bmatrix},
        \matp_2=\begin{bmatrix}
            \mati_{r_2} & \mato \\\mato&\mato
        \end{bmatrix},
    \end{equation*}
    如果\(r_1\neq r_2\),则\(\matp_2\)不能由\(\matp_1\)经过初等变换得到。
\end{lemma}

\begin{theorem}
    任一非零矩阵经有限次初等变换可化为标准形
    \begin{equation*}
        \begin{bmatrix}
            \mati_r & \mato \\\mato&\mato
        \end{bmatrix}
    \end{equation*}
    且对于给定的矩阵,其标准形中\(r\)的值是唯一确定的。
\end{theorem}

\begin{infer}
    矩阵经过初等变换后其秩不变。
\end{infer}

\begin{theorem}
    设\(m\times n\)矩阵\(\mata\)和\(\matb\),下列三个命题等价:
    \begin{enumerate}
        \item 矩阵\(\mata\)与\(\matb\)的秩相等;
        \item 矩阵\(\mata\)与\(\matb\)具有相同的标准形;
        \item 矩阵\(\mata\)经有限次初等变换可化为矩阵\(\matb\)。
    \end{enumerate}
\end{theorem}

\begin{theorem}
    \(\rank{\mata^\top}=\rank{\mata}\)。
\end{theorem}

\begin{theorem}
    任一非零矩阵只经初等行变换可化为(最简)阶梯型矩阵。
\end{theorem}

\begin{infer}
    非零矩阵的秩等于其(最简)阶梯型矩阵中主元列的个数。
\end{infer}

\section{线性方程组}

\subsection{横看线性方程组}

\subsubsection*{齐次线性方程组的解}

\end{document}