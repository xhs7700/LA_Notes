\documentclass{ctexart}
\usepackage{anyfontsize}
\usepackage{hyperref}
\usepackage{graphicx}
\usepackage{amsmath,amsthm}

\author{夏海淞}
\title{线性代数笔记}
\date{\today}

\ctexset { section = { name={第,章} } }
\ctexset { section = { number={\chinese {section}} } }

\input{headers.tex}

\begin{document}
\maketitle
\tableofcontents
\section{矩阵}

\subsection{矩阵和向量的基本概念}

\subsubsection*{矩阵的基本概念}

\subsubsection*{向量的基本概念}

\subsection{矩阵与向量的运算}

\subsubsection*{矩阵(向量)的线性运算}

\subsubsection*{向量的内积与矩阵的乘法}

\begin{property}
    设矩阵\(\mata_{m\times l},\matb_{l\times n}\),则\(\mypar{\mata\matb}^\top=\matb^\top\mata^\top\)。
\end{property}

\begin{property}
    设矩阵\(\mata_{m\times n},\matb_{n\times m}\),则\(\trace{\mata\matb}=\trace{\matb\mata}\)。
\end{property}

\begin{property}
    设\(\mata\)为\(m\times n\)矩阵,\(\mati_m\)和\(\mati_n\)分别为\(m\)阶单位阵和\(n\)阶单位阵,则\(\mati_m\mata=\mata\mati_n=\mata\)。
\end{property}

\begin{property}
    设\(\mata=\diag{a_1,a_2,\dots,a_n}\),\(\matb=\diag{b_1,b_2,\dots,b_n}\),则\(\mata\matb=\matb\mata=\diag{a_1b_1,a_2b_2,\dots,a_nb_n}\)。
\end{property}

\begin{property}
    设\(\mata\)与\(\matb\)均为\(n\)阶上(下)三角阵,则\(\mata\matb\)是上(下)三角阵。
\end{property}

\subsubsection*{方阵的幂}

\subsection{分块矩阵及其运算}

\subsubsection*{分块矩阵}

\subsubsection*{分块矩阵的基本运算}

\subsection{矩阵的初等变换与秩}

\subsubsection*{矩阵的初等变换}

\begin{theorem}
    对矩阵\(\mata_{m\times n}\)作一次初等行变换,相当于在\(\mata_{m\times n}\)的左边乘上一个相应的\(m\)阶初等矩阵;对矩阵\(\mata_{m\times n}\)作一次初等列变换,相当于在\(\mata_{m\times n}\)的右边乘上一个相应的\(n\)阶初等矩阵。
\end{theorem}

\subsubsection*{矩阵的标准形与秩}

\begin{lemma}
    设两个\(m\times n\)矩阵的标准形如下所示
    \begin{equation*}
        \matp_1=\begin{bmatrix}
            \mati_{r_1} & \mato \\\mato&\mato
        \end{bmatrix},
        \matp_2=\begin{bmatrix}
            \mati_{r_2} & \mato \\\mato&\mato
        \end{bmatrix},
    \end{equation*}
    如果\(r_1\neq r_2\),则\(\matp_2\)不能由\(\matp_1\)经过初等变换得到。
\end{lemma}

\begin{theorem}
    任一非零矩阵经有限次初等变换可化为标准形
    \begin{equation*}
        \begin{bmatrix}
            \mati_r & \mato \\\mato&\mato
        \end{bmatrix}
    \end{equation*}
    且对于给定的矩阵,其标准形中\(r\)的值是唯一确定的。
\end{theorem}

\begin{infer}
    矩阵经过初等变换后其秩不变。
\end{infer}

\begin{theorem}
    设\(m\times n\)矩阵\(\mata\)和\(\matb\),下列三个命题等价:
    \begin{enumerate}
        \item 矩阵\(\mata\)与\(\matb\)的秩相等;
        \item 矩阵\(\mata\)与\(\matb\)具有相同的标准形;
        \item 矩阵\(\mata\)经有限次初等变换可化为矩阵\(\matb\)。
    \end{enumerate}
\end{theorem}

\begin{theorem}
    \(\rank{\mata^\top}=\rank{\mata}\)。
\end{theorem}

\begin{theorem}
    任一非零矩阵只经初等行变换可化为(最简)阶梯型矩阵。
\end{theorem}

\begin{infer}
    非零矩阵的秩等于其(最简)阶梯型矩阵中主元列的个数。
\end{infer}

\section{线性方程组}

\subsection{横看线性方程组}

\subsubsection*{齐次线性方程组的解}

\end{document}