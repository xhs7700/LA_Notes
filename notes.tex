\documentclass{ctexart}
\usepackage{anyfontsize}
\usepackage{hyperref}
\usepackage{graphicx}
\usepackage{amsmath,amsthm}

\author{夏海淞}
\title{线性代数笔记}
\date{\today}

\ctexset { section = { name={第,章} } }
\ctexset { section = { number={\chinese {section}} } }

\input{headers.tex}

\begin{document}
\maketitle
\tableofcontents
\section{矩阵}

\subsection{矩阵和向量的基本概念}

\subsubsection*{矩阵的基本概念}

\subsubsection*{向量的基本概念}

\subsection{矩阵与向量的运算}

\subsubsection*{矩阵(向量)的线性运算}

\subsubsection*{向量的内积与矩阵的乘法}

\begin{property}
    设矩阵\(\mata_{m\times l},\matb_{l\times n}\),则\(\mypar{\mata\matb}^\top=\matb^\top\mata^\top\)。
\end{property}

\begin{property}
    设矩阵\(\mata_{m\times n},\matb_{n\times m}\),则\(\trace{\mata\matb}=\trace{\matb\mata}\)。
\end{property}

\begin{property}
    设\(\mata\)为\(m\times n\)矩阵,\(\mati_m\)和\(\mati_n\)分别为\(m\)阶单位阵和\(n\)阶单位阵,则\(\mati_m\mata=\mata\mati_n=\mata\)。
\end{property}

\begin{property}
    设\(\mata=\diag{a_1,a_2,\dots,a_n}\),\(\matb=\diag{b_1,b_2,\dots,b_n}\),则\(\mata\matb=\matb\mata=\diag{a_1b_1,a_2b_2,\dots,a_nb_n}\)。
\end{property}

\begin{property}
    设\(\mata\)与\(\matb\)均为\(n\)阶上(下)三角阵,则\(\mata\matb\)是上(下)三角阵。
\end{property}

\subsubsection*{方阵的幂}

\subsection{分块矩阵及其运算}

\subsubsection*{分块矩阵}

\subsubsection*{分块矩阵的基本运算}

\subsection{矩阵的初等变换与秩}

\subsubsection*{矩阵的初等变换}

\begin{theorem}
    对矩阵\(\mata_{m\times n}\)作一次初等行变换,相当于在\(\mata_{m\times n}\)的左边乘上一个相应的\(m\)阶初等矩阵;对矩阵\(\mata_{m\times n}\)作一次初等列变换,相当于在\(\mata_{m\times n}\)的右边乘上一个相应的\(n\)阶初等矩阵。
\end{theorem}

\subsubsection*{矩阵的标准形与秩}

\begin{lemma}
    设两个\(m\times n\)矩阵的标准形如下所示
    \begin{equation*}
        \matp_1=\begin{bmatrix}
            \mati_{r_1} & \mato \\\mato&\mato
        \end{bmatrix},
        \matp_2=\begin{bmatrix}
            \mati_{r_2} & \mato \\\mato&\mato
        \end{bmatrix},
    \end{equation*}
    如果\(r_1\neq r_2\),则\(\matp_2\)不能由\(\matp_1\)经过初等变换得到。
\end{lemma}

\begin{theorem}
    任一非零矩阵经有限次初等变换可化为标准形
    \begin{equation*}
        \begin{bmatrix}
            \mati_r & \mato \\\mato&\mato
        \end{bmatrix}
    \end{equation*}
    且对于给定的矩阵,其标准形中\(r\)的值是唯一确定的。
\end{theorem}

\begin{infer}
    矩阵经过初等变换后其秩不变。
\end{infer}

\begin{theorem}
    设\(m\times n\)矩阵\(\mata\)和\(\matb\),下列三个命题等价:
    \begin{enumerate}
        \item 矩阵\(\mata\)与\(\matb\)的秩相等;
        \item 矩阵\(\mata\)与\(\matb\)具有相同的标准形;
        \item 矩阵\(\mata\)经有限次初等变换可化为矩阵\(\matb\)。
    \end{enumerate}
\end{theorem}

\begin{theorem}
    \(\rank{\mata^\top}=\rank{\mata}\)。
\end{theorem}

\begin{theorem}
    任一非零矩阵只经初等行变换可化为(最简)阶梯型矩阵。
\end{theorem}

\begin{infer}
    非零矩阵的秩等于其(最简)阶梯型矩阵中主元列的个数。
\end{infer}

\section{线性方程组}

\subsection{横看线性方程组}

\subsubsection*{齐次线性方程组的解}

\begin{theorem}
    齐次线性方程组只有零解当且仅当系数矩阵的秩等于未知量的个数;齐次线性方程组有非零解当且仅当系数矩阵的秩小于未知量的个数。
\end{theorem}

\subsubsection*{非齐次线性方程组的解}

\begin{theorem}
    线性方程组有解的充要条件是系数矩阵的秩等于增广矩阵的秩。
\end{theorem}

\subsection{纵看线性方程组}

\subsubsection*{线性相关与向量组的秩}

\begin{theorem}
    向量组\(\enums{\vecal}{s}(s\ge2)\)线性相关的充要条件是该向量组中至少有一个向量可由其余向量线性表出。
\end{theorem}

\begin{theorem}
    向量组\(\enums{\vecal}{s}\)线性相关当且仅当矩阵\(\mata=\mat{\enums{\vecal}{s}}\)的秩\(\rank{\mata}<s\);向量组\(\enums{\vecal}{s}\)线性无关当且仅当\(\rank{\mata}=s\)。
\end{theorem}

\begin{infer}
    \(s(s>n)\)个\(n\)元向量线性相关。
\end{infer}

\begin{lemma}
    设向量组\(\enums{\vecal}{s}\)线性无关,而向量组\(\enums{\vecal}{s},\vecal\)线性相关,则\(\vecal\)可由\(\enums{\vecal}{s}\)线性表出,且线性表示式唯一。
\end{lemma}

\begin{theorem}
    向量组中任一向量都可由该向量组的极大线性无关组线性表出。
\end{theorem}

\begin{lemma}
    设向量组\(\enums{\vecal}{r}\)可由向量组\(\enums{\vecbeta}{s}\)线性表出,若向量组\(\enums{\vecal}{r}\)线性无关,则\(r\le s\)。
\end{lemma}

\begin{theorem}
    一个向量组的任意两个极大线性无关组可以相互线性表出,且所含向量的个数相等。
\end{theorem}

\begin{theorem}
    矩阵的秩等于列(行)秩。
\end{theorem}

\begin{theorem}
    \(\rank{\mata\matb}\le\min\setof{\rank{\mata},\rank{\matb}}\),其中\(\mata,\matb\)分别为\(m\times l,l\times n\)矩阵。
\end{theorem}

\begin{infer}
    设\(\vecal\)为\(n\)元非零行向量,则\(\rank{\vecal\vecal^\top}=1\)。
\end{infer}

\subsubsection*{齐次线性方程组的基础解系}

\begin{theorem}
    设\(\enums{\vecx}{t}\)均为齐次线性方程组\(\mata\vecx=\veczero\)的解,则\(\enums{\vecx}{t}\)的线性组合也是\(\mata\vecx=\veczero\)的解。
\end{theorem}

\begin{definition}[基础解系]
    设\(\enums{\vecx}{t}\)是齐次线性方程组\(\mata\vecx=\veczero\)的一组解向量,若其满足条件:
    \begin{enumerate}
        \item 线性无关;
        \item 齐次线性方程组的任一解向量都能由\(\enums{\vecx}{t}\)线性表出,
    \end{enumerate}
    则称\(\enums{\vecx}{t}\)为齐次线性方程组\(\mata\vecx=\veczero\)的\emph{基础解系}。
\end{definition}

\begin{theorem}
    设\(\rank{\mata_{m\times n}}=r\),若\(\mata\vecx=\veczero\)有非零解,则该齐次线性方程组有基础解系,且基础解系含有\(n-r\)个解。
\end{theorem}

\begin{theorem}
    设\(\mata\)为\(m\times n\)矩阵,\(\vecb\)为\(m\)元列向量,则
    \begin{enumerate}
        \item \(\rank{\mata\mata^\top}=\rank{\mata}=\rank{\mata^\top}\);
        \item 线性方程组\(\mata^\top\mata\vecx=\mata\vecb\)一定有解。
    \end{enumerate}
\end{theorem}

\subsubsection*{非齐次线性方程组解的结构}

\begin{theorem}
    设\(\vecx_0\)是非齐次线性方程组\(\mata\vecx=\vecb\)的某个特定解(简称为特解),\(\vecy\)是相应齐次线性方程组\(\mata\vecx=\veczero\)的通解,则非齐次线性方程组\(\mata\vecx=\vecb\)的通解为\(\vecx=\vecx_0+\vecy\)。
\end{theorem}

\subsection{逆矩阵}

\subsubsection*{可逆矩阵的定义与性质}

\begin{theorem}
    设\(\mata\)是\(n\)阶方阵,下述若干命题等价:
    \begin{enumerate}
        \item \(\rank{\mata}=n\);
        \item 存在\(n\)阶方阵\(\matb\)使\(\mata\matb=\mati\);
        \item 存在\(n\)阶方阵\(\matc\)使\(\matc\mata=\mati\);
        \item \(\mata\)的列向量组线性无关;
        \item \(\mata\)的行向量组线性无关;
        \item \(\mata\vecx=\veczero\)只有零解;
        \item \(\mata\)可经过初等变换化为标准形\(\mati\)。
    \end{enumerate}
\end{theorem}

\begin{property}
    设\(\mata\)是\(n\)阶可逆矩阵,则
    \begin{enumerate}
        \item \(\inv{\mypar{\inv{\mata}}}=\mata\);
        \item \(\inv{\mypar{k\mata}}=\inv{k}\inv{\mata}\);
        \item \(\inv{\mypar{\mata^\top}}=\mypar{\inv{\mata}}^\top\);
        \item \(\inv{\mypar{\mata\matb}}=\inv{\matb}\inv{\mata}\),其中\(\matb\)也是\(n\)阶可逆矩阵。
    \end{enumerate}
\end{property}

\begin{infer}
    设\(\enums{\mata}{k}\)是\(n\)阶可逆矩阵,则\(\mata_1\mata_2\cdots\mata_k\)也可逆,且\(\inv{\mypar{\mata_1\mata_2\cdots\mata_k}}=\inv{\mata_k}\cdots\inv{\mata_2}\inv{\mata_1}\)。
\end{infer}

\begin{property}
    \begin{enumerate}
        \item 若对角阵可逆,则其逆矩阵仍为对角阵;
        \item 若对称阵可逆,则其逆矩阵仍为对称阵。
    \end{enumerate}
\end{property}

\begin{theorem}
    初等阵可逆,且初等阵的逆矩阵仍为初等阵。
\end{theorem}

\subsubsection*{用初等变换求逆矩阵}

\begin{theorem}
    任一秩为\(r\)的非零矩阵\(\mata_{m\times n}\),必存在\(m\)阶可逆矩阵\(\matp\)及\(n\)阶可逆矩阵\(\matq\),使得
    \begin{equation*}
        \matp\mata\matq=\begin{bmatrix}
            \mati_r & \mato \\\mato&\mato
        \end{bmatrix}
    \end{equation*}
\end{theorem}

\begin{infer}
    任一秩为\(r\)的非零矩阵\(\mata_{m\times n}\),必存在\(m\)阶可逆矩阵\(\tilde{\matp}\)及\(n\)阶可逆矩阵\(\tilde{\matq}\),使得
    \begin{equation*}
        \mata=\tilde{\matp}\begin{bmatrix}
            \mati_r & \mato \\\mato&\mato
        \end{bmatrix}\tilde{\matq}
    \end{equation*}
\end{infer}

\begin{theorem}
    方阵\(\mata\)可逆当且仅当下列条件之一成立:
    \begin{enumerate}
        \item 方阵\(\mata\)的标准形为单位阵;
        \item 方阵\(\mata\)可表示成若干初等阵的乘积;
        \item 方阵\(\mata\)仅经初等行变换可化为单位阵;
        \item 方阵\(\mata\)仅经初等列变换可化为单位阵。
    \end{enumerate}
\end{theorem}

\subsubsection*{正交阵}

\begin{theorem}
    设矩阵\(\mata\)为\(n\)阶方阵,则下列命题等价:
    \begin{enumerate}
        \item \(\mata\)为正交阵;
        \item \(\mata^\top\mata=\mati\);
        \item \(\mata\mata^\top=\mati\);
        \item 将\(\mata\)按列分块为\(\mata=\mat{\enums{\vecal}{n}}\),有\({\vecal_i}^\top\vecal_j=\delta_{ij}\)\((i,j=1,2,\dots,n)\),其中\begin{equation*}
                  \delta_{ij}=\begin{cases}
                      1 & i=j \\0&i\neq j
                  \end{cases}
              \end{equation*}
        \item 将\(\mata\)按行分块为\begin{equation*}
                  \mata=\begin{bmatrix}
                      \vecbeta_1 \\\vecbeta_2\\\vdots\\\vecbeta_n
                  \end{bmatrix}
              \end{equation*}有\(\vecbeta_i{\vecbeta_j}^\top=\delta_{ij}\)\(i,j=1,2,\dots,n\)。
    \end{enumerate}
\end{theorem}

\begin{property}
    \begin{enumerate}
        \item 正交阵的积仍为正交阵;
        \item 正交阵的逆矩阵(即其转置矩阵)仍为正交阵。
    \end{enumerate}
\end{property}

\section{行列式}

\subsection{行列式的定义}

\begin{definition}[行列式]
    设\(n\)阶方阵
    \begin{equation*}
        \mata=\sqmat{bmatrix}{a}{n}
    \end{equation*}
    则方阵\(\mata\)的\emph{行列式}可定义为
    \begin{equation*}
        \sqmat{vmatrix}{a}{n}=\sum_{i_1i_2\cdots i_n}(-1)^{\tau\mypar{i_1i_2\cdots i_n}}a_{i_11}a_{i_22}\cdots a_{i_nn}
    \end{equation*}
\end{definition}

\subsubsection*{逆序数}

\begin{theorem}
    任一排列经过一次对换(排列中某两个数的位置互换而其余的数不变),其逆序数的奇偶性改变。
\end{theorem}

\subsubsection*{行列式的定义}

\subsection{行列式的性质}

\begin{property}
    设\(\mata\)是\(n\)阶方阵,则\(\det{\mata^\top}=\det{\mata}\)。
\end{property}

\begin{property}
    设\(n\)阶方阵
    \begin{align*}
        \mata & =\begin{bmatrix}
                     a_{11}        & a_{12}        & \cdots & a_{1n}        \\
                     \vdots        & \vdots        &        & \vdots        \\
                     b_{t1}+c_{t1} & b_{t2}+c_{t2} & \cdots & b_{tn}+c_{tn} \\
                     \vdots        & \vdots        &        & \vdots        \\
                     a_{n1}        & a_{n2}        & \cdots & a_{nn}
                 \end{bmatrix} \\
        \matb & =\begin{bmatrix}
                     a_{11} & a_{12} & \cdots & a_{1n} \\
                     \vdots & \vdots &        & \vdots \\
                     b_{t1} & b_{t2} & \cdots & b_{tn} \\
                     \vdots & \vdots &        & \vdots \\
                     a_{n1} & a_{n2} & \cdots & a_{nn}
                 \end{bmatrix},\matc=\begin{bmatrix}
                                         a_{11} & a_{12} & \cdots & a_{1n} \\
                                         \vdots & \vdots &        & \vdots \\
                                         c_{t1} & c_{t2} & \cdots & c_{tn} \\
                                         \vdots & \vdots &        & \vdots \\
                                         a_{n1} & a_{n2} & \cdots & a_{nn}
                                     \end{bmatrix}
    \end{align*}
    则\(\det{\mata}=\det{\matb}+\det{\matc}\),即方阵的行列式具有分行(列)相加性。
\end{property}

\begin{property}
    方阵的任意两行(列)互换,其行列式的值只改变正负号。
\end{property}

\begin{infer}
    若方阵中有两行(列)对应元素相等,则其行列式为\(0\)。
\end{infer}

\begin{property}
    设\(n\)阶方阵
    \begin{equation*}
        \mata=\begin{bmatrix}
            a_{11} & a_{12} & \cdots & a_{1n} \\
            \vdots & \vdots &        & \vdots \\
            a_{i1} & a_{i2} & \cdots & a_{in} \\
            \vdots & \vdots &        & \vdots \\
            a_{n1} & a_{n2} & \cdots & a_{nn}
        \end{bmatrix},\matb=\begin{bmatrix}
            a_{11}  & a_{12}  & \cdots & a_{1n}  \\
            \vdots  & \vdots  &        & \vdots  \\
            ka_{i1} & ka_{i2} & \cdots & ka_{in} \\
            \vdots  & \vdots  &        & \vdots  \\
            a_{n1}  & a_{n2}  & \cdots & a_{nn}
        \end{bmatrix}
    \end{equation*}
    则\(\det{\matb}=k\det{\mata}\),即将\(\mata\)的第\(i\)行(列)乘以常数\(k\)所得新矩阵的行列式为\(k\det{\mata}\)。
\end{property}

\begin{infer}
    设\(\mata\)是\(n\)阶方阵,\(\det{k\mata}=k^n\det{\mata}\)。
\end{infer}

\begin{infer}
    若方阵某一行(列)的元素全为\(0\),则其行列式为\(0\)。
\end{infer}

\begin{infer}
    若方阵中有两行(列)的元素对应成比例,则其行列式为\(0\)。
\end{infer}

\begin{property}
    设\(n\)阶方阵
    \begin{equation*}
        \mata=\begin{bmatrix}
            a_{11} & a_{12} & \cdots & a_{1n} \\
            \vdots & \vdots &        & \vdots \\
            a_{i1} & a_{i2} & \cdots & a_{in} \\
            \vdots & \vdots &        & \vdots \\
            a_{j1} & a_{j2} & \cdots & a_{jn} \\
            \vdots & \vdots &        & \vdots \\
            a_{n1} & a_{n2} & \cdots & a_{nn}
        \end{bmatrix}
    \end{equation*}
    \begin{equation*}
        \matb=\begin{bmatrix}
            a_{11}         & a_{12}         & \cdots & a_{1n}         \\
            \vdots         & \vdots         &        & \vdots         \\
            a_{i1}+ka_{j1} & a_{i2}+ka_{j2} & \cdots & a_{in}+ka_{jn} \\
            \vdots         & \vdots         &        & \vdots         \\
            a_{j1}         & a_{j2}         & \cdots & a_{jn}         \\
            \vdots         & \vdots         &        & \vdots         \\
            a_{n1}         & a_{n2}         & \cdots & a_{nn}
        \end{bmatrix}
    \end{equation*}
    则\(\det{\matb}=\det{\mata}\),即将\(\mata\)的第\(j\)行(列)乘以常数\(k\)后加到第\(i\)行(列)所得矩阵的行列式不变。
\end{property}

\begin{infer}
    3种初等变换对应初等矩阵\(\mate_{ij},\mate_j(k),\mate_{ij}(k)\)的行列式分别为\(\det{\mate_{ij}}=-1\),\(\det{\mate_i\mypar{k}}=k\neq0\),\(\det{\mate_{ij}\mypar{k}}=1\)。
\end{infer}

\begin{infer}
    设\(\mate\)是某种\(n\)阶初等矩阵,\(\mata\)是\(n\)阶方阵,则\(\det{\mate\mata}\)=\(\det{\mata\mate}=\det{\mata}\det{\mate}\)。
\end{infer}

\begin{theorem}
    方阵\(\mata\)可逆的充要条件是\(\det{\mata}\neq0\)。
\end{theorem}

\begin{infer}
    方阵\(\mata\)不可逆的充要条件是\(\det{\mata}=0\)。
\end{infer}

\begin{property}
    设\(\mata\)和\(\matb\)都是\(n\)阶方阵,则\(\det{\mata\matb}=\det{\mata}\det{\matb}\),即两个\(n\)阶方阵乘积的行列式等于两个方阵行列式的乘积。
\end{property}

\begin{infer}
    设\(\enums{\mata}{k}\)是\(n\)阶方阵,则
    \begin{equation*}
        \det{\mata_1\mata_2\cdots\mata_k}=\det{\mata_1}\det{\mata_2}\cdots\det{\mata_k}
    \end{equation*}
\end{infer}

\begin{property}
    设\(n\)阶方阵\(\mata\)可逆,则\(\det{\inv{\mata}}=\inv{\det{\mata}}\)。
\end{property}

\subsection{伴随矩阵与行列式按行(列)展开}

\subsubsection*{伴随矩阵}

\begin{definition}[代数余子式]
    将\(n\)阶方阵\(\mata=\mat{a_{ij}}_{n\times n}\)的第\(i\)行第\(j\)列划去后,所得的\(n-1\)阶子矩阵的行列式记作\(M_{ij}\),则称\((-1)^{i+j}M_{ij}\)为元素\(a_{ij}\)的\emph{代数余子式},记作\(A_{ij}=(-1)^{i+j}M_{ij}\)。
\end{definition}

\begin{definition}[伴随矩阵]
    设\(A_{ij}\)是矩阵\(\mata=\mat{a_{ij}}_{n\times n}\)中元素\(a_{ij}\)的代数余子式,则称矩阵
    \begin{equation*}
        \sqmat{bmatrix}{A}{n}
    \end{equation*}
    为矩阵\(\mata\)的\emph{伴随矩阵},记作\(\mata^*\)或\(\adj{\mata}\)。
\end{definition}

\begin{lemma}
    设
    \begin{equation*}
        \mata=\begin{bmatrix}
            a_{11} & 0      & \cdots & 0      \\
            a_{21} & a_{22} & \cdots & a_{2n} \\
            \vdots & \vdots &        & \vdots \\
            a_{n1} & a_{n2} & \cdots & a_{nn}
        \end{bmatrix}
    \end{equation*}
    则\(\det{\mata}=a_{11}M_{11}\),其中\(M_{11}\)是将方阵\(\mata\)的第一行第一列划去后所得的\(n-1\)阶子矩阵的行列式:
    \begin{equation*}
        M_{11}=\begin{vmatrix}
            a_{22} & \cdots & a_{2n} \\
            \vdots &        & \vdots \\
            a_{n2} & \cdots & a_{nn}
        \end{vmatrix}
    \end{equation*}
\end{lemma}

\begin{theorem}
    设方阵\(\mata\)的伴随矩阵为\(\mata^*\),则\(\mata\mata^*=\mata^*\mata=\det{\mata}\mati\),其中\(\mati\)为单位阵。
\end{theorem}

\begin{infer}
    设方阵\(\mata\)可逆,则\(\inv{\mata}=\inv{\det{\mata}}\mata^*\),其中\(\mata^*\)是\(\mata\)的伴随矩阵。
\end{infer}

\subsubsection*{行列式按行(列)展开}

\subsubsection*{Cramer法则}

\begin{theorem}[Cramer法则]
    设\(n\)阶方阵\(\mata\)可逆,\(\vecb\)是\(n\)元列向量,则线性方程组\(\mata\vecx=\vecb\)的唯一解是\(\vecx=\mat{\enums{x}{n}}^\top\),其中
    \begin{equation*}
        x_1=\frac{\det{\mata_1}}{\det{\mata}},x_2=\frac{\det{\mata_2}}{\det{\mata}},\dots,x_n=\frac{\det{\mata_n}}{\det{\mata}}
    \end{equation*}
    这里\(\mata_j(j=1,2,\dots,n)\)是将\(\mata\)中第\(j\)列的元素\(a_{1j},a_{2j},\dots,a_{nj}\)分别换成向量\(\vecb\)中的元素\(\enums{b}{n}\)所得的矩阵。
\end{theorem}

\subsection{行列式与矩阵的秩}

\begin{theorem}
    设\(\mata\)为\(m\times n\)矩阵,\(\rank{\mata}=r\)的充要条件是矩阵\(\mata\)有一个\(r\)阶子矩阵的行列式不为零,而\(\mata\)的所有\(r+1\)阶子矩阵(如果存在的话)的行列式均为零。
\end{theorem}

\section{线性空间与线性变换}

\subsection{线性空间}

\begin{definition}[线性空间]
    设\(V\)是一个非空集合,\(\field\)是一个数域,若下述条件均成立,则称集合\(V\)为数域\(\field\)上的\emph{线性空间}。
    \begin{enumerate}
        \item 在\(V\)中的任意两个元素\(\vecal,\vecbeta\)之间定义一个对应法则,使得\(\vecal,\vecbeta\)在\(V\)中有唯一确定的元素\(\vecgamma\)与它们对应,称这个对应法则为\emph{加法},称\(\vecgamma\)为\(\vecal\)和\(\vecbeta\)的\emph{和},记作\(\vecgamma=\vecal+\vecbeta\)。
        \item 在数域\(\field\)中的任意元素\(k\)与集合\(V\)中的任意元素\(\vecal\)之间定义一个对应法则,使得\(k,\vecal\)在\(V\)中有唯一确定的元素\(\vecgamma\)与它们对应,称这个对应法则为\emph{数量乘法},简称\emph{数乘},称\(\vecgamma\)为\(k,\vecal\)的\emph{数量乘积},记作\(\vecgamma=k\vecal\)。
        \item 对于上述定义的加法与数乘满足八项规则:\begin{enumerate}
                  \item \(\vecal+\vecbeta=\vecbeta+\vecal\);
                  \item \(\mypar{\vecal+\vecbeta}+\vecdelta=\vecal+\mypar{\vecbeta+\vecdelta}\);
                  \item 在\(V\)中存在零元,记作\(\veczero\),对于\(V\)中任一元素\(\vecal\),使得\(\vecal+\veczero=\vecal\)成立;
                  \item 对于\(V\)中任一元素\(\vecal\),在\(V\)中存在相应的负元,记作\(\mypar{-\vecal}\),使得\(\vecal+\mypar{-\vecal}=\veczero\)成立;
                  \item \(1\cdot\vecal=\vecal\);
                  \item \(k\mypar{l\vecal}=\mypar{kl}\vecal\);
                  \item \(k\mypar{\vecal+\vecbeta}=k\vecal+k\vecbeta\);
                  \item \(\mypar{k+l}\vecal=k\vecal+l\vecal\),
              \end{enumerate}
    \end{enumerate}
    其中\(\vecal,\vecbeta,\vecdelta\)为集合\(V\)中任意元素,\(k,l\)为数域\(\field\)中的任意数。
\end{definition}

\begin{property}
    \begin{enumerate}
        \item 线性空间\(V\)的零元是唯一的;
        \item 线性空间\(V\)中每个元素的负元是唯一的。
    \end{enumerate}
\end{property}

\subsection{基}

\subsubsection*{基和坐标}

\begin{definition}[基]
    设\(V\)是线性空间,如果在\(V\)中存在\(n\)个线性无关向量\(\enums{\epsilon}{n}\),使得\(V\)中任一向量\(\vecal\)均可由\(\enums{\epsilon}{n}\)线性表出,则称向量组\(\enums{\epsilon}{n}\)为线性空间\(V\)的一组\emph{基},\(n\)称为线性空间\(V\)的\emph{维数},记\(\dim{V}=n\),\(V\)称为\(n\)维线性空间。如果不存在有限个向量构成\(V\)的一组基,则\(V\)称为无限维线性空间。
\end{definition}

\begin{definition}[坐标]
    设向量组\(\enums{\epsilon}{n}\)是\(n\)维线性空间\(V\)的一组基,\(\vecal\)是\(V\)中任一向量,若
    \begin{equation*}
        \vecal=x_1\epsilon_1+x_2\epsilon_2+\cdots+x_n\epsilon_n
    \end{equation*}
    则称\(\enums{x}{n}\)为向量\(\vecal\)在基\(\enums{\epsilon}{n}\)下的\emph{坐标}。
\end{definition}

\begin{theorem}
    任意向量在给定基下的坐标唯一。
\end{theorem}

\subsubsection*{过渡矩阵}

\begin{definition}[过渡矩阵]
    设\(\enums{\veceps}{n}\)和\(\enums{\veceta}{n}\)是\(n\)维线性空间\(V\)的两组基,若\(\mat{\enums{\veceta}{n}}=\mat{\enums{\veceps}{n}}\matm\),则称\(n\)阶方阵\(\matm\)为从基\(\enums{\veceps}{n}\)到基\(\enums{\veceta}{n}\)的\emph{过渡矩阵}。
\end{definition}

\begin{theorem}
    过渡矩阵是可逆的。
\end{theorem}

\begin{theorem}[坐标变换公式]
    设\(V\)是\(n\)维线性空间,\(\matm\)是由基\(\enums{\veceps}{n}\)到基\(\enums{\veceta}{n}\)的过渡矩阵,\(V\)中的向量\(\vecal\)在基\(\enums{\veceps}{n}\)和基\(\enums{\veceta}{n}\)下的坐标分别为\(\mat{\enums{x}{n}}^\top\)和\(\mat{\enums{x'}{n}}^\top\),则
    \begin{equation*}
        \begin{bmatrix}
            x_1 \\x_2\\\vdots\\x_n
        \end{bmatrix}=\matm\begin{bmatrix}
            x'_1 \\x'_2\\\vdots\\x'_n
        \end{bmatrix}
    \end{equation*}
\end{theorem}

\subsection{子空间}

\subsubsection*{子空间的定义}

\begin{theorem}
    设\(V\)是数域\(\field\)上的线性空间,\(W\)是\(V\)的一个非空子集,若满足条件:
    \begin{enumerate}
        \item 如果\(\vecal,\vecbeta\in W\),则\(\vecal+\vecbeta\in W\);
        \item 如果\(\vecal\in W,\lambda\in\field\),则\(\lambda\vecal\in W\),
    \end{enumerate}
    则\(W\)为\(V\)的一个子空间。
\end{theorem}

\begin{theorem}
    向量组\(\enums{\vecal}{l}\)张成的子空间\(\myspan{\enums{\vecal}{l}}\)的维数等于向量组\(\enums{\vecal}{l}\)的秩。
\end{theorem}

\subsubsection*{零空间与列空间}

\begin{theorem}
    齐次线性方程组的解集\(\setof{\vecx\in\rea^n|\mata\vecx=\veczero}\)构成线性空间\(\rea^n\)的一个子空间;非齐次线性方程组的解集\(\setof{\vecx\in\rea^n|\mata\vecx=\vecb,\vecb\neq0}\)不构成子空间。
\end{theorem}

\begin{definition}[零空间]
    齐次线性方程组\(\mata\vecx=\veczero\)的解集称为该齐次线性方程组的解空间,或称为矩阵\(\mata\)的\emph{零空间},记作\(\nullsp{\mata}\)。
\end{definition}

\begin{definition}[列空间]
    矩阵\(\mata\)的列向量张成的子空间称为矩阵\(\mata\)的\emph{列空间},记作\(\colsp{\mata}\)。
\end{definition}

\begin{theorem}
    线性方程组\(\mata\vecx=\vecb\)有解的充要条件是\(\vecb\in\colsp{\mata}\)。
\end{theorem}

\begin{theorem}
    设\(m\times n\)矩阵\(\mata\)的秩\(\rank{\mata}=r\),则
    \begin{enumerate}
        \item 矩阵\(\mata\)的列空间\(\colsp{\mata}=\setof{\sum_{j=1}^n\lambda_j\col{\mata}{j}|\lambda_j\in\rea}\)的维数\(\dim{\colsp{\mata}}=r\);
        \item 矩阵\(\mata\)的零空间\(\nullsp{\mata}=\setof{\vecx\in\rea^n|\mata\vecx=\veczero}\)的维数\(\dim{\nullsp{\mata}}=n-r\)。
    \end{enumerate}
\end{theorem}

\subsubsection*{子空间的交与和}

\begin{theorem}
    设\(W_1\)与\(W_2\)是线性空间\(V\)的子空间,则\(W_1\cap W_2\)与\(W_1+W_2\)均为\(V\)的子空间。
\end{theorem}

\begin{theorem}[维数公式]
    设\(W_1\)与\(W_2\)是线性空间\(V\)的子空间,则
    \begin{equation*}
        \dim W_1+\dim W_2=\dim\mypar{W_1+W_2}+\dim\mypar{W_1\cap W_2}
    \end{equation*}
\end{theorem}

\begin{definition}[补空间]
    设\(W_1\)与\(W_2\)是线性空间\(V\)的子空间,如果
    \begin{equation*}
        W_1\cap W_2=\setof{\veczero},W_1+W_2=V
    \end{equation*}
    则称\(W_2\)是\(W_1\)关于线性空间\(V\)的\emph{补空间}。
\end{definition}

\begin{theorem}[补空间存在性定理]
    设\(W_1\)是\(n\)维线性空间\(V\)的子空间,则存在\(W_1\)关于线性空间\(V\)的补空间。
\end{theorem}

\subsection{内积空间}

\subsubsection*{内积}

\begin{definition}[内积]
    设\(V\)是实数域\(\rea\)上的一个线性空间,如果二元运算\(\indot{\vecal}{\vecbeta}\)满足以下条件:
    \begin{enumerate}
        \item \(\indot{\vecal}{\vecbeta}=\indot{\vecbeta}{\vecal}\);
        \item \(\indot{k\vecal}{\vecbeta}=k\indot{\vecal}{\vecbeta}\);
        \item \(\indot{\vecal+\vecbeta}{\vecgamma}=\indot{\vecal}{\vecgamma}+\indot{\vecbeta}{\vecgamma}\);
        \item \(\indot{\vecal}{\vecal}\ge0\),且\(\indot{\vecal}{\vecal}=0\)当且仅当\(\vecal\)为零元。
    \end{enumerate}
    其中\(\vecal,\vecbeta,\vecgamma\in V\),\(k\in\rea\),则称二元运算\(\indot{\vecal}{\vecbeta}\)为\(\vecal\)与\(\vecbeta\)的\emph{内积}。引入内积的线性空间称为内积空间。
\end{definition}

\begin{property}
    \begin{equation*}
        \indot{\sum_{i=1}^nk_i\vecal_i}{\sum_{j=1}^ml_j\vecbeta_j}=\sum_{i=1}^n\sum_{j=1}^mk_il_j\indot{\vecal_i}{\vecbeta_j}
    \end{equation*}
\end{property}

\begin{property}[Cauchy-Schwarz不等式]
    设\(\vecal,\vecbeta\)是内积空间\(V\)上的向量,则恒有
    \begin{equation*}
        \indot{\vecal}{\vecbeta}^2\le\indot{\vecal}{\vecal}\indot{\vecbeta}{\vecbeta}
    \end{equation*}
\end{property}

\begin{definition}[模]
    设\(\vecal\)是内积空间\(V\)中的一个向量,则\(\sqrt{\indot{\vecal}{\vecal}}\)称为向量\(\vecal\)的\emph{模}或者范数,记为\(\Abs{\vecal}\)。
\end{definition}

\begin{definition}[正交]
    设\(V\)是一个内积空间,\(\vecal,\vecbeta\in V\),如果\(\indot{\vecal}{\vecbeta}=0\),则称\(\vecal\)与\(\vecbeta\)\emph{正交},记为\(\vecal\bot\vecbeta\)。
\end{definition}

\begin{theorem}
    设\(\enums{\vecal}{l}(l\le n)\)是\(n\)维内积空间\(V\)中的一个正交向量组,则\(\enums{\vecal}{l}\)线性无关。
\end{theorem}

\begin{theorem}
    设\(\enums{\veceps}{n}\)是\(n\)维内积空间\(V\)的一个标准正交基,\(\vecal\in V\),\(\vecbeta\in V\),\(\vecal\)与\(\vecbeta\)在该标准正交基下的坐标向量分别为\(\vecx=\mat{\enums{x}{n}}^\top\)和\(\vecy=\mat{\enums{y}{n}}^\top\),则\(\vecal\)与\(\vecbeta\)的内积\(\indot{\vecal}{\vecbeta}=\vecx^\top\vecy\)。
\end{theorem}

\subsubsection*{正交投影与最小二乘解}

\subsubsection*{Schimidt正交化}

\subsubsection*{正交补空间}

\begin{theorem}
    若\(W_1\bot W_2\),则\(W_1\cap W_2=\setof{\veczero}\)。
\end{theorem}

\begin{definition}[正交补空间]
    设\(W_1,W_2\)是内积空间\(V\)的两个子空间,如果\(W_1\bot W_2\),且\(W_1+W_2=V\),则称\(W_2\)是\(W_1\)的\emph{正交补空间},简称正交补,记作\(W_2=\ortcom{W_1}\)。
\end{definition}

\begin{infer}
    \(\colsp{\mata}\)是\(\nullsp{\mata^\top}\)的正交补空间,即\(\ortcom{\colsp{\mata}}=\nullsp{\mata^\top}\)。
\end{infer}

\subsection{线性变换}

\subsubsection*{线性映射与线性变换}

\begin{property}
    设\(T\)是线性空间\(V\)中的线性变换,则\(T\mypar{\veczero}=\veczero\)。
\end{property}

\begin{property}
    设\(T\)是线性空间\(V\)中的线性变换,\(\enums{\vecal}{m}\)是\(V\)中的\(m\)个向量,\(\vecx=\mat{\enums{x}{m}}^\top\),则
    \begin{equation*}
        T\mypar{\mat{\enums{\vecal}{m}\vecx}}=T\mypar{\enums{\vecal}{m}}\vecx
    \end{equation*}
\end{property}

\begin{theorem}
    设\(\enums{\veceps}{n}\)是\(n\)维线性空间\(V\)的一组基,对于\(V\)中任意\(n\)个向量\(\enums{\vecal}{n}\),存在唯一的线性变换\(T\)使得
    \begin{equation*}
        T(\veceps_i)=\vecal_i(i=1,2,\dots,n)
    \end{equation*}
\end{theorem}

\subsubsection*{线性变换的表示矩阵}

\begin{definition}[表示矩阵]
    设\(\enums{\veceps}{n}\)是\(n\)维线性空间\(V\)的一组基,\(T\)是线性空间\(V\)的一个线性变换,基的像可以表示为
    \begin{align*}
        T\mypar{\enums{\veceps}{n}} & =\mat{T\mypar{\veceps_1},T\mypar{\veceps_2},\dots,T\mypar{\veceps_n}} \\
                                    & =\mat{\enums{\veceps}{n}}\sqmat{bmatrix}{\alpha}{n}
    \end{align*}
    其中矩阵
    \begin{equation*}
        \sqmat{bmatrix}{\alpha}{n}
    \end{equation*}
    称为线性变换\(T\)在基\(\enums{\veceps}{n}\)下的\emph{表示矩阵}。
\end{definition}

\begin{theorem}
    在给定基下,线性变换与其表示矩阵是一一对应的。
\end{theorem}

\begin{theorem}
    设线性变换\(T\)在基\(\enums{\veceps}{n}\)下的表示矩阵是\(\mata\),向量\(\xi\)在基\(\enums{\veceps}{n}\)下的坐标为\(\vecx=\mat{\enums{x}{n}}^\top\),则\(T\mypar{\xi}\)在基\(\enums{\veceps}{n}\)下的坐标为\(\mata\vecx\)。
\end{theorem}

\begin{theorem}
    设线性变换\(T\)在线性空间\(V\)的两组基\(\enums{\veceps}{n}\)和\(\enums{\veceta}{n}\)下的表示矩阵分别是\(\mata\)和\(\matb\),从基\(\enums{\veceps}{n}\)到基\(\enums{\veceta}{n}\)的过渡矩阵为\(\matm\),则\(\matb=\inv{\matm}\mata\matm\)。
\end{theorem}

\begin{definition}[相似]
    对于\(n\)阶矩阵\(\mata\)和\(\matb\),若存在一个\(n\)阶可逆矩阵\(\matp\),使得\(\inv{\matp}\mata\matp=\matb\),则称矩阵\(\mata\)与\(\matb\)\emph{相似},记作\(\mata\sim\matb\)。
\end{definition}

\subsubsection*{正交变换}

\begin{theorem}
    正交变换保持向量的内积不变。
\end{theorem}

\end{document}